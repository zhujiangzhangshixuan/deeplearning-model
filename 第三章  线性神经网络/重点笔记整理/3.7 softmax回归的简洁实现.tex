\documentclass[UTF8]{ctexart}
\usepackage[textwidth=444bp,vmargin=2.5cm]{geometry}
%\usepackage{abstract} %生成摘要使用的宏包
\usepackage[colorlinks,linkcolor=black,anchorcolor=black,citecolor=black]{hyperref} %“colorlinks”的意思是将超链接以颜色来标识,而并非使用默认的方框来标识。linkcolor,anchorcolor, citecolor分别表示用来标识link, anchor, cite等各种链接的颜色。此处我们均设为黑色。
\usepackage{fancyhdr} %插入页脚的宏包
\usepackage{graphicx}%图片宏包
\usepackage{float}
\graphicspath{{./figure/}}
\DeclareGraphicsExtensions{.pdf,.jpeg,.png,.jpg}%在添加图片后只需要图片的名字,而不需要拓展名
\usepackage{subfigure}%并列插入图片
\usepackage{amsmath}%公式宏包
\usepackage[T1]{fontenc}% 统一修改正文和数学字体为Adobe Utopia, 这个字体和Times有些像
\usepackage{newtxtext, newtxmath}  %两种使用Times New Roman 字体的方法
\linespread{1.5}%设置行间距为1.5倍行距
\renewcommand{\abstractname}{\Large\textbf{摘要}}
\usepackage{appendix}%加入附录需使用appendix宏包
\usepackage{listings}%插入代码
\usepackage{xcolor}

\usepackage{booktabs}
\usepackage{longtable}
\usepackage{array}
\usepackage{multirow}
\usepackage{makecell}
\usepackage{bm}
\usepackage{pythonhighlight}
\usepackage{ctex}
\usepackage[linesnumbered,ruled]{algorithm2e}
\ctexset{
	% 修改 section。
	section={   
		name={,、},
		number={\chinese{section}}
	}
}
\ctexset{
	% 修改 section。
	section={   
		name={,、},
		number={\chinese{section}},
		aftername=\hspace{1pt}
	}
}
\usepackage{enumitem}
\usepackage{tabularx}
%\usepackage{setspace}%使用间距宏包

\begin{document}    %文档的开始,一定要有文档的结束,才能生效
	\setlength{\abovedisplayskip}{2pt}
	\setlength{\belowdisplayskip}{2pt}
	\setlength{\abovedisplayshortskip}{2pt}
	\setlength{\belowdisplayshortskip}{2pt}
	%------------------------标题-------------------------
	%\begin{titlepage}%使页码跳过这页
	\begin{center}
		\heiti\zihao{3}\textbf{3.7 \, softmax回归的简洁实现重点摘录与练习解答} %标题3号加粗
		\vspace{2ex}
	\end{center}
	
	
	%\end{titlepage}
	%-------------------------正文部分-------------------------
	%修改页眉页脚
	\pagestyle{fancy}
	\lhead{}
	\chead{}
	\rhead{}
	%\lfoot{}
	\cfoot{\thepage}
	%\rfoot{} %空格即表示空白
	\renewcommand{\headrulewidth}{0pt}
	\renewcommand{\footrulewidth}{0pt} %设置页眉页脚分割线的宽度,如果为0pt,则不显示线条
	
	
	%正文中引用到参考文献的地方\cite{01} \cite{02}
	%-------------------------附录部分-------------------------
	% 使用\begin{appendices} \end{appendices} 或者直接用\appendix
	\appendix
	%代码格式设置,代码的设置与具体的编程语言有关,比赛时上网搜素即可
	\definecolor{dkgreen}{rgb}{0,0.6,0}
	\definecolor{gray}{rgb}{0.5,0.5,0.5}
	\definecolor{mauve}{rgb}{0.58,0,0.82}
	\definecolor{mydarkblue}{RGB}{0, 0, 128} % 示例:深蓝色,RGB值为(0, 0, 128) 
	\definecolor{codegreen}{rgb}{0,0.6,0}
	\definecolor{codegray}{rgb}{0.5,0.5,0.5}
	\definecolor{codepurple}{rgb}{0.58,0,0.82}
	\definecolor{backcolour}{rgb}{0.95,0.95,0.92}
	\lstset{ %
		language=Python,                % the language of the code
		basicstyle=\footnotesize,           % the size of the fonts that are used for the code
		numbers=left,                   % where to put the line-numbers
		numberstyle=\tiny\color{gray},  % the style that is used for the line-numbers
		stepnumber=2,                   % the step between two line-numbers. If it's 1, each line 
		% will be numbered
		numbersep=5pt,                  % how far the line-numbers are from the code
		backgroundcolor=\color{white},      % choose the background color. You must add \usepackage{color}
		showspaces=false,               % show spaces adding particular underscores
		showstringspaces=false,         % underline spaces within strings
		showtabs=false,                 % show tabs within strings adding particular underscores
		frame=single,                   % adds a frame around the code
		rulecolor=\color{black},        % if not set, the frame-color may be changed on line-breaks within not-black text (e.g. commens (green here))
		tabsize=2,                      % sets default tabsize to 2 spaces
		captionpos=b,                   % sets the caption-position to bottom
		breaklines=true,                % sets automatic line breaking
		breakatwhitespace=false,        % sets if automatic breaks should only happen at whitespace
		title=\lstname,                   % show the filename of files included with \lstinputlisting;
		% also try caption instead of title
		keywordstyle=\color{blue},          % keyword style
		commentstyle=\color{codegreen},       % comment style
		stringstyle=\color{codepurple},         % string literal style
		escapeinside={\%*}{*)},            % if you want to add LaTeX within your code
		morekeywords={*,...}               % if you want to add more keywords to the set
	}
	
	\textbf{(1) softmax 函数的改进}
	
	在前面的交叉熵计算中,从数学上讲,这是合理的。然而,从数值计算的角度来看,指数可能会造成数值稳定性问题。
	
	根据前文定义,softmax函数 $\hat{y}_j = \frac{\exp(o_j)}{\sum_k \exp(o_k)}$,其中 $\hat{y}_j$ 是预测的概率分布,$o_j$ 是未规范化的预测 $\bm{o}$ 的第 $j$ 个元素。如果 $o_k$ 中的一些数值非常大,那么 $\exp(o_k)$ 可能大于数据类型容许的最大数字,即上溢(overflow)。这将使分母或分子变为 inf(无穷大),最后得到的是 0、inf 或 nan(不是数字)的 $\hat{y}_j$。在这些情况下,我们无法得到一个明确定义的交叉熵值。
	
	解决这个问题的一个技巧是:在继续 softmax 计算之前,先从所有 $o_k$ 中减去 $\max(o_k)$。这里可以看到每个 $o_k$ 按常数进行的移动不会改变 softmax 的返回值:
	\begin{eqnarray*}
		\hat{y}_j &=& \frac{\exp(o_j - \max(o_k)) \exp(\max(o_k))}{\sum_k \exp(o_k - \max(o_k)) \exp(\max(o_k))}\\
		&=& \frac{\exp(o_j - \max(o_k))}{\sum_k \exp(o_k - \max(o_k))}.
	\end{eqnarray*}

	在减法和规范化步骤之后,可能有些 $o_j - \max(o_k)$ 具有较大的负值。由于精度受限,$\exp(o_j - \max(o_k))$ 将有接近零的值,即下溢(underflow)。这些值可能会四舍五入为零,使 $\hat{y}_j$ 为零,并且使得 $\log(\hat{y}_j)$ 的值为 -inf。反向传播几步后,可能会出现可怕的 nan 结果。
	
	尽管我们要计算指数函数,但我们最终在计算交叉熵损失时会取它们的对数。通过将 softmax 和交叉熵结合在一起,可以避免反向传播过程中可能会困扰我们的数值稳定性问题。如下面的等式所示,我们避免计算 $\exp(o_j - \max(o_k))$,而可以直接使用 $o_j - \max(o_k)$,因为 $\log(\exp(\cdot))$ 被抵消了。
	\begin{eqnarray*}
		\log(\hat{y}_j) &=& \log\left(\frac{\exp(o_j - \max(o_k))}{\sum_k \exp(o_k - \max(o_k))}\right)\\
		&=& \log\left(\exp(o_j - \max(o_k))\right) - \log\left(\sum_k \exp(o_k - \max(o_k))\right)\\
		&=& o_j - \max(o_k) - \log\left(\sum_k \exp(o_k - \max(o_k))\right).
	\end{eqnarray*}
	
	
	我们也希望保留传统的 softmax 函数,以备我们需要评估通过模型输出的概率。但是,我们没有将 softmax 概率传递到损失函数中,而是在交叉熵损失函数中传递未规范化的预测,并同时计算 softmax 及其对数。
	
	\newpage
	\textbf{(2) 问题解答}
	
	\textbf{2、增加迭代周期的数量。为什么测试精度会在一段时间后降低?我们怎么解决这个问题?}
	
	\noindent \textbf{解}:因为样本复杂度小于模型复杂度,出现过拟合导致的。
	
	随着迭代周期的数量增加,模型不断去接近样本规律,但到了某一迭代次数后,模型表达能力过剩,会去学习一些只能满足训练样本的非共性特征,即过拟合,从而降低模型的泛化能力,导致测试精度降低。
	
	可以通过以下方法解决:
	
	1)数据增强(Data Augmentation):通过对训练数据进行各种随机变换(如旋转、平移、缩放、翻转等),扩增训练数据的多样性,可以降低过拟合风险。
	
	2)正则化(Regularization):通过在损失函数中引入正则化项(如L1正则化、L2正则化),限制模型参数的大小,防止模型过于复杂而出现过拟合。
	
	3)早停(Early Stopping):在训练过程中监控验证集的性能,当验证集性能不再提升时,停止训练,避免过拟合。
	
	4)Dropout:通过在训练过程中随机将一部分神经元的输出置为0,来减少神经元之间的依赖关系,降低过拟合。
	
	5)模型复杂度调整:减少模型的复杂度,可以通过减少网络层数、减少神经元个数等方式,降低过拟合风险。
	
	6)数据集分割:合理划分训练集、验证集和测试集,用于模型的训练、调参和评估,以确保模型在未知数据上的泛化能力
	
\end{document}