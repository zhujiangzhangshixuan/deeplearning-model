\documentclass[UTF8]{ctexart}
\usepackage[textwidth=444bp,vmargin=2.5cm]{geometry}
%\usepackage{abstract} %生成摘要使用的宏包
\usepackage[colorlinks,linkcolor=black,anchorcolor=black,citecolor=black]{hyperref} %“colorlinks”的意思是将超链接以颜色来标识,而并非使用默认的方框来标识。linkcolor,anchorcolor, citecolor分别表示用来标识link, anchor, cite等各种链接的颜色。此处我们均设为黑色。
\usepackage{fancyhdr} %插入页脚的宏包
\usepackage{graphicx}%图片宏包
\usepackage{float}
\graphicspath{{./figure/}}
\DeclareGraphicsExtensions{.pdf,.jpeg,.png,.jpg}%在添加图片后只需要图片的名字,而不需要拓展名
\usepackage{subfigure}%并列插入图片
\usepackage{amsmath}%公式宏包
\usepackage[T1]{fontenc}% 统一修改正文和数学字体为Adobe Utopia, 这个字体和Times有些像
\usepackage{newtxtext, newtxmath}  %两种使用Times New Roman 字体的方法
\linespread{1.5}%设置行间距为1.5倍行距
\renewcommand{\abstractname}{\Large\textbf{摘要}}
\usepackage{appendix}%加入附录需使用appendix宏包
\usepackage{listings}%插入代码
\usepackage{xcolor}

\usepackage{booktabs}
\usepackage{longtable}
\usepackage{array}
\usepackage{multirow}
\usepackage{makecell}
\usepackage{bm}
\usepackage{pythonhighlight}
\usepackage{ctex}
\usepackage[linesnumbered,ruled]{algorithm2e}
\ctexset{
	% 修改 section。
	section={   
		name={,、},
		number={\chinese{section}}
	}
}
\ctexset{
	% 修改 section。
	section={   
		name={,、},
		number={\chinese{section}},
		aftername=\hspace{1pt}
	}
}
\usepackage{enumitem}
\usepackage{tabularx}
%\usepackage{setspace}%使用间距宏包

\begin{document}    %文档的开始,一定要有文档的结束,才能生效
	\setlength{\abovedisplayskip}{2pt}
	\setlength{\belowdisplayskip}{2pt}
	\setlength{\abovedisplayshortskip}{2pt}
	\setlength{\belowdisplayshortskip}{2pt}
	%------------------------标题-------------------------
	%\begin{titlepage}%使页码跳过这页
	\begin{center}
		\heiti\zihao{3}\textbf{3.5 \, 图像分类数据集重点摘录与练习解答} %标题3号加粗
		\vspace{2ex}
	\end{center}
	
	
	%\end{titlepage}
	%-------------------------正文部分-------------------------
	%修改页眉页脚
	\pagestyle{fancy}
	\lhead{}
	\chead{}
	\rhead{}
	%\lfoot{}
	\cfoot{\thepage}
	%\rfoot{} %空格即表示空白
	\renewcommand{\headrulewidth}{0pt}
	\renewcommand{\footrulewidth}{0pt} %设置页眉页脚分割线的宽度,如果为0pt,则不显示线条

	
	\textbf{问题解答}
	
	\textbf{1、减少batch\_size(如减少到1)是否会影响读取性能?}
	
	\noindent \textbf{解}:较大的 batch\_size 可以利用并行化操作来提高数据加载的效率。在读取数据时,可以同时加载多个样本到内存中,以减少磁盘读取数据的时间。而当 batch\_size 减少到1时,无法利用并行化操作,每次只能加载一个样本,从而导致数据加载的效率降低。
	
	此外,较小的 batch\_size 也会带来更多的数据加载次数。数据加载在计算机中属于 I/O 操作,通常是相对较慢的。因此,较小的 batch\_size 会导致更频繁的数据加载操作,增加了整体的读取时间。
	
	然而,较小的 batch\_size 也具有一些优点,比如更快的模型更新和更精确的梯度估计。因此,在选择 batch\_size 时,需要综合考虑性能和模型训练的需求,权衡不同因素。\\
	
	\textbf{2、数据迭代器的性能非常重要。当前的实现足够快吗?探索各种选择来改进它}
	
	\noindent \textbf{解}:
	\begin{itemize}
		\item 使用 pin\_memory=True:在创建数据加载器时,设置 pin\_memory=True 来将数据加载到 GPU 内存中,利用GPU的并行性来加快数据传输速度。
		\item 使用多个 get\_dataloader\_workers:通过增加 get\_dataloader\_workers 参数的值,可以使用多个子进程或线程来并行读取数据,从而加快数据加载的速度。可以根据计算机的硬件配置和数据加载的需求,选择合适的 get\_dataloader\_workers 值。\\
	\end{itemize}
	
	
	\textbf{5、查阅框架的在线API文档。还有哪些其他数据集可用?}
	
	%正文中引用到参考文献的地方\cite{01} \cite{02}
	%-------------------------附录部分-------------------------
	% 使用\begin{appendices} \end{appendices} 或者直接用\appendix
	\appendix
	%代码格式设置,代码的设置与具体的编程语言有关,比赛时上网搜素即可
	\definecolor{dkgreen}{rgb}{0,0.6,0}
	\definecolor{gray}{rgb}{0.5,0.5,0.5}
	\definecolor{mauve}{rgb}{0.58,0,0.82}
	\definecolor{mydarkblue}{RGB}{0, 0, 128} % 示例:深蓝色,RGB值为(0, 0, 128) 
	\definecolor{codegreen}{rgb}{0,0.6,0}
	\definecolor{codegray}{rgb}{0.5,0.5,0.5}
	\definecolor{codepurple}{rgb}{0.58,0,0.82}
	\definecolor{backcolour}{rgb}{0.95,0.95,0.92}
	\lstset{ %
		language=Python,                % the language of the code
		basicstyle=\footnotesize,           % the size of the fonts that are used for the code
		numbers=left,                   % where to put the line-numbers
		numberstyle=\tiny\color{gray},  % the style that is used for the line-numbers
		stepnumber=2,                   % the step between two line-numbers. If it's 1, each line 
		% will be numbered
		numbersep=5pt,                  % how far the line-numbers are from the code
		backgroundcolor=\color{white},      % choose the background color. You must add \usepackage{color}
		showspaces=false,               % show spaces adding particular underscores
		showstringspaces=false,         % underline spaces within strings
		showtabs=false,                 % show tabs within strings adding particular underscores
		frame=single,                   % adds a frame around the code
		rulecolor=\color{black},        % if not set, the frame-color may be changed on line-breaks within not-black text (e.g. commens (green here))
		tabsize=2,                      % sets default tabsize to 2 spaces
		captionpos=b,                   % sets the caption-position to bottom
		breaklines=true,                % sets automatic line breaking
		breakatwhitespace=false,        % sets if automatic breaks should only happen at whitespace
		title=\lstname,                   % show the filename of files included with \lstinputlisting;
		% also try caption instead of title
		keywordstyle=\color{blue},          % keyword style
		commentstyle=\color{codegreen},       % comment style
		stringstyle=\color{codepurple},         % string literal style
		escapeinside={\%*}{*)},            % if you want to add LaTeX within your code
		morekeywords={*,...}               % if you want to add more keywords to the set
	}
	
	
	\noindent \textbf{解:} 
	
	\begin{lstlisting}
		dir(torchvision.datasets)
	\end{lstlisting}
	
	
	
\end{document}