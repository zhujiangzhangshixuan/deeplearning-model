\documentclass[UTF8]{ctexart}
\usepackage[textwidth=444bp,vmargin=2.5cm]{geometry}
%\usepackage{abstract} %生成摘要使用的宏包
\usepackage[colorlinks,linkcolor=black,anchorcolor=black,citecolor=black]{hyperref} %“colorlinks”的意思是将超链接以颜色来标识,而并非使用默认的方框来标识。linkcolor,anchorcolor, citecolor分别表示用来标识link, anchor, cite等各种链接的颜色。此处我们均设为黑色。
\usepackage{fancyhdr} %插入页脚的宏包
\usepackage{graphicx}%图片宏包
\usepackage{float}
\graphicspath{{./figure/}}
\DeclareGraphicsExtensions{.pdf,.jpeg,.png,.jpg}%在添加图片后只需要图片的名字,而不需要拓展名
\usepackage{subfigure}%并列插入图片
\usepackage{amsmath}%公式宏包
\usepackage[T1]{fontenc}% 统一修改正文和数学字体为Adobe Utopia, 这个字体和Times有些像
\usepackage{newtxtext, newtxmath}  %两种使用Times New Roman 字体的方法
\linespread{1.5}%设置行间距为1.5倍行距
\renewcommand{\abstractname}{\Large\textbf{摘要}}
\usepackage{appendix}%加入附录需使用appendix宏包
\usepackage{listings}%插入代码
\usepackage{xcolor}

\usepackage{booktabs}
\usepackage{longtable}
\usepackage{array}
\usepackage{multirow}
\usepackage{makecell}
\usepackage{bm}
\usepackage{pythonhighlight}
\usepackage{ctex}
\usepackage[linesnumbered,ruled]{algorithm2e}
\ctexset{
	% 修改 section。
	section={   
		name={,、},
		number={\chinese{section}}
	}
}
\ctexset{
	% 修改 section。
	section={   
		name={,、},
		number={\chinese{section}},
		aftername=\hspace{1pt}
	}
}
\usepackage{enumitem}
\usepackage{tabularx}
%\usepackage{setspace}%使用间距宏包

\begin{document}    %文档的开始,一定要有文档的结束,才能生效
	\setlength{\abovedisplayskip}{2pt}
	\setlength{\belowdisplayskip}{2pt}
	\setlength{\abovedisplayshortskip}{2pt}
	\setlength{\belowdisplayshortskip}{2pt}
	%------------------------标题-------------------------
	%\begin{titlepage}%使页码跳过这页
	\begin{center}
		\heiti\zihao{3}\textbf{3.2 \, 线性回归从零开始重点摘录与练习解答} %标题3号加粗
		\vspace{2ex}
	\end{center}
	
	
	%\end{titlepage}
	%-------------------------正文部分-------------------------
	%修改页眉页脚
	\pagestyle{fancy}
	\lhead{}
	\chead{}
	\rhead{}
	%\lfoot{}
	\cfoot{\thepage}
	%\rfoot{} %空格即表示空白
	\renewcommand{\headrulewidth}{0pt}
	\renewcommand{\footrulewidth}{0pt} %设置页眉页脚分割线的宽度,如果为0pt,则不显示线条
	
	\textbf{(1) 参数更新}
	
	我们将执行以下循环:
	\begin{itemize}
		\item 初始化参数
		\item 重复以下训练,直到完成
		
		计算梯度
		\[
		\mathbf{g} \leftarrow \partial_{(\mathbf{w},b)} \frac{1}{|\mathcal{B}|} \sum_{i \in \mathcal{B}} l(\mathbf{x}^{(i)}, y^{(i)}, \mathbf{w}, b)
		\]
		
		更新参数
		\[
		(\mathbf{w}, b) \leftarrow (\mathbf{w}, b) - \eta \mathbf{g}
		\]
	\end{itemize}
	
	\textbf{(2) 问题解答}
	
	\textbf{1、如果我们将权重初始化为零,会发生什么。算法仍然有效吗?}
	
	\noindent \textbf{解}:在线性回归中,由于只有一层神经网络,且SGD过程中,梯度求导后结果与参数本身无关,而是取决于输入 $\bm{X}$ 和 $y$,因此,可以将权重初始化为0,算法仍然有效,在代码部分有实践。
		
	但是,在多层神经网络中,如果将权重初始化为0,或者其他统一的常量,会导致后面迭代的权重更新相同,并且神经网络中的激活单元的值相同,输出的梯度也相等,导致对称性问题,无法进行独立学习,找到最优解。
	
	\textbf{4、计算二阶导数时可能会遇到什么问题?这些问题可以如何解决?}
	
	\noindent \textbf{解}:二阶导数包含了更多关于损失函数曲率的信息,因此在某些情况下,计算二阶导数可能有助于更快地收敛和更准确的更新。
		
	以下是计算二阶导数时可能会遇到的问题,以及可能的解决方法:
	\begin{enumerate}
		\item 计算复杂度高:计算Hessian矩阵需要更多计算资源和时间,尤其是大规模数据和复杂模型
		
		解决方法: 通常可以使用近似方法来估计二阶导数,例如L-BFGS(Limited-memory Broyden-Fletcher-Goldfarb-Shanno)等优化算法。这些方法在一定程度上降低了计算成本,同时仍能提供较好的优化效果。
		\item 存储需求大:Hessian矩阵存储需求随着参数数量的增加而增加,可能导致内存不足的问题
		
		解决方法: 使用一些高效的矩阵近似方法,如块对角近似(block-diagonal approximation)或采样Hessian近似,来减少存储需求。
		\item 数值不稳定性:在计算Hessian矩阵时,可能会遇到数值不稳定,导致数值误差累积,影响优化结果。
		
		解决方法: 使用数值稳定的计算方法,例如通过添加小的正则化项来避免矩阵的奇异性。另外,选择合适的优化算法和学习率调度也可以帮助稳定优化过程。
		\item 局部极小值和鞍点: 在高维空间中,存在许多局部极小值和鞍点,这可能导致Hessian矩阵的谱值较小,使得计算二阶导数的结果不稳定。
		
		解决方法: 使用正则化技术、随机性优化方法(如随机梯度牛顿法)或基于自适应学习率的算法,可以帮助逃离局部极小值和鞍点。
	\end{enumerate}
	
	\textbf{5、为什么在 squared\_loss 函数中需要使用 reshape 函数?}
	
	\noindent \textbf{解:} 因为 y\_hat 和 y 的元素个数相同,但 shape 不一定相同(虽然在本节中二者 shape 一致),为了保证计算时不出错,故使用 reshape 函数将二者的 shape 统一。
	
	
	%正文中引用到参考文献的地方\cite{01} \cite{02}
	%-------------------------附录部分-------------------------
	% 使用\begin{appendices} \end{appendices} 或者直接用\appendix
	\appendix
	%代码格式设置,代码的设置与具体的编程语言有关,比赛时上网搜素即可
	\definecolor{dkgreen}{rgb}{0,0.6,0}
	\definecolor{gray}{rgb}{0.5,0.5,0.5}
	\definecolor{mauve}{rgb}{0.58,0,0.82}
	\definecolor{mydarkblue}{RGB}{0, 0, 128} % 示例:深蓝色,RGB值为(0, 0, 128) 
	\definecolor{codegreen}{rgb}{0,0.6,0}
	\definecolor{codegray}{rgb}{0.5,0.5,0.5}
	\definecolor{codepurple}{rgb}{0.58,0,0.82}
	\definecolor{backcolour}{rgb}{0.95,0.95,0.92}
	\lstset{ %
		language=Python,                % the language of the code
		basicstyle=\footnotesize,           % the size of the fonts that are used for the code
		numbers=left,                   % where to put the line-numbers
		numberstyle=\tiny\color{gray},  % the style that is used for the line-numbers
		stepnumber=2,                   % the step between two line-numbers. If it's 1, each line 
		% will be numbered
		numbersep=5pt,                  % how far the line-numbers are from the code
		backgroundcolor=\color{white},      % choose the background color. You must add \usepackage{color}
		showspaces=false,               % show spaces adding particular underscores
		showstringspaces=false,         % underline spaces within strings
		showtabs=false,                 % show tabs within strings adding particular underscores
		frame=single,                   % adds a frame around the code
		rulecolor=\color{black},        % if not set, the frame-color may be changed on line-breaks within not-black text (e.g. commens (green here))
		tabsize=2,                      % sets default tabsize to 2 spaces
		captionpos=b,                   % sets the caption-position to bottom
		breaklines=true,                % sets automatic line breaking
		breakatwhitespace=false,        % sets if automatic breaks should only happen at whitespace
		title=\lstname,                   % show the filename of files included with \lstinputlisting;
		% also try caption instead of title
		keywordstyle=\color{blue},          % keyword style
		commentstyle=\color{codegreen},       % comment style
		stringstyle=\color{codepurple},         % string literal style
		escapeinside={\%*}{*)},            % if you want to add LaTeX within your code
		morekeywords={*,...}               % if you want to add more keywords to the set
	}
	
	
	
	
	
\end{document}