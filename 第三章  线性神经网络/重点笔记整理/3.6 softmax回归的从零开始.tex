\documentclass[UTF8]{ctexart}
\usepackage[textwidth=444bp,vmargin=2.5cm]{geometry}
%\usepackage{abstract} %生成摘要使用的宏包
\usepackage[colorlinks,linkcolor=black,anchorcolor=black,citecolor=black]{hyperref} %“colorlinks”的意思是将超链接以颜色来标识,而并非使用默认的方框来标识。linkcolor,anchorcolor, citecolor分别表示用来标识link, anchor, cite等各种链接的颜色。此处我们均设为黑色。
\usepackage{fancyhdr} %插入页脚的宏包
\usepackage{graphicx}%图片宏包
\usepackage{float}
\graphicspath{{./figure/}}
\DeclareGraphicsExtensions{.pdf,.jpeg,.png,.jpg}%在添加图片后只需要图片的名字,而不需要拓展名
\usepackage{subfigure}%并列插入图片
\usepackage{amsmath}%公式宏包
\usepackage[T1]{fontenc}% 统一修改正文和数学字体为Adobe Utopia, 这个字体和Times有些像
\usepackage{newtxtext, newtxmath}  %两种使用Times New Roman 字体的方法
\linespread{1.5}%设置行间距为1.5倍行距
\renewcommand{\abstractname}{\Large\textbf{摘要}}
\usepackage{appendix}%加入附录需使用appendix宏包
\usepackage{listings}%插入代码
\usepackage{xcolor}

\usepackage{booktabs}
\usepackage{longtable}
\usepackage{array}
\usepackage{multirow}
\usepackage{makecell}
\usepackage{bm}
\usepackage{pythonhighlight}
\usepackage{ctex}
\usepackage[linesnumbered,ruled]{algorithm2e}
\ctexset{
	% 修改 section。
	section={   
		name={,、},
		number={\chinese{section}}
	}
}
\ctexset{
	% 修改 section。
	section={   
		name={,、},
		number={\chinese{section}},
		aftername=\hspace{1pt}
	}
}
\usepackage{enumitem}
\usepackage{tabularx}
%\usepackage{setspace}%使用间距宏包

\begin{document}    %文档的开始,一定要有文档的结束,才能生效
	\setlength{\abovedisplayskip}{2pt}
	\setlength{\belowdisplayskip}{2pt}
	\setlength{\abovedisplayshortskip}{2pt}
	\setlength{\belowdisplayshortskip}{2pt}
	%------------------------标题-------------------------
	%\begin{titlepage}%使页码跳过这页
	\begin{center}
		\heiti\zihao{3}\textbf{3.6 \, softmax回归的从零开始重点摘录与练习解答} %标题3号加粗
		\vspace{2ex}
	\end{center}
	
	
	%\end{titlepage}
	%-------------------------正文部分-------------------------
	%修改页眉页脚
	\pagestyle{fancy}
	\lhead{}
	\chead{}
	\rhead{}
	%\lfoot{}
	\cfoot{\thepage}
	%\rfoot{} %空格即表示空白
	\renewcommand{\headrulewidth}{0pt}
	\renewcommand{\footrulewidth}{0pt} %设置页眉页脚分割线的宽度,如果为0pt,则不显示线条
	
	
	%正文中引用到参考文献的地方\cite{01} \cite{02}
	%-------------------------附录部分-------------------------
	% 使用\begin{appendices} \end{appendices} 或者直接用\appendix
	\appendix
	%代码格式设置,代码的设置与具体的编程语言有关,比赛时上网搜素即可
	\definecolor{dkgreen}{rgb}{0,0.6,0}
	\definecolor{gray}{rgb}{0.5,0.5,0.5}
	\definecolor{mauve}{rgb}{0.58,0,0.82}
	\definecolor{mydarkblue}{RGB}{0, 0, 128} % 示例:深蓝色,RGB值为(0, 0, 128) 
	\definecolor{codegreen}{rgb}{0,0.6,0}
	\definecolor{codegray}{rgb}{0.5,0.5,0.5}
	\definecolor{codepurple}{rgb}{0.58,0,0.82}
	\definecolor{backcolour}{rgb}{0.95,0.95,0.92}
	\lstset{ %
		language=Python,                % the language of the code
		basicstyle=\footnotesize,           % the size of the fonts that are used for the code
		numbers=left,                   % where to put the line-numbers
		numberstyle=\tiny\color{gray},  % the style that is used for the line-numbers
		stepnumber=2,                   % the step between two line-numbers. If it's 1, each line 
		% will be numbered
		numbersep=5pt,                  % how far the line-numbers are from the code
		backgroundcolor=\color{white},      % choose the background color. You must add \usepackage{color}
		showspaces=false,               % show spaces adding particular underscores
		showstringspaces=false,         % underline spaces within strings
		showtabs=false,                 % show tabs within strings adding particular underscores
		frame=single,                   % adds a frame around the code
		rulecolor=\color{black},        % if not set, the frame-color may be changed on line-breaks within not-black text (e.g. commens (green here))
		tabsize=2,                      % sets default tabsize to 2 spaces
		captionpos=b,                   % sets the caption-position to bottom
		breaklines=true,                % sets automatic line breaking
		breakatwhitespace=false,        % sets if automatic breaks should only happen at whitespace
		title=\lstname,                   % show the filename of files included with \lstinputlisting;
		% also try caption instead of title
		keywordstyle=\color{blue},          % keyword style
		commentstyle=\color{codegreen},       % comment style
		stringstyle=\color{codepurple},         % string literal style
		escapeinside={\%*}{*)},            % if you want to add LaTeX within your code
		morekeywords={*,...}               % if you want to add more keywords to the set
	}
	
	
	\textbf{问题解答}
	
	\textbf{1、本节直接实现了基于数学定义 softmax 运算的 softmax 函数。这可能会导致什么问题?提示:尝试计算 $e^{50}$ 的大小}
	
	\noindent \textbf{解}:当输入向量中的元素较大时,exp 计算可能会出现数值溢出 (无穷大,NAN) 的问题;
	
	而当输入向量中的元素较小时,exp 计算可能会出现数值下溢(分母为0)的问题。\\
	
	\textbf{2、本节中的函数cross\_entropy是根据交叉熵损失函数的定义实现的。它可能有什么问题?提示:考虑对数的定义域。}
	
	\noindent \textbf{解}:当预测标签向量中的元素接近0时,$\log$ 计算可能会出现数值溢出 (无穷大,NAN) 的问题。\\
	
	\textbf{3、请想一个解决方案来解决上述两个问题。}
	
	\noindent \textbf{解}:对于第一问题,处理方法如下:
	
	采用减去最大值操作使得 $\exp$ 计算的最大输入为 0,排除了数值溢出的可能性,即令
	\[
	\bm{Z} = \bm{O} - \max(o_i)
	\]
	
	则
	\[
	\text{softmax}(\bm{Z}) = \frac{\exp(z_j)}{\sum_{k=1}^q \exp(z_k)}
	\]
	
	同时使得分母 $\geq 1$ ($\exp(0) = 1$),排除了分母数值下溢导致被零除的可能。
	
	对于第二个问题的处理方法如下:
	
	采用减去最大值操作使得交叉熵损失计算的最大输入为 $(\min(o_j) - \max(o_i))$,不会超出数据类型容许的最大数字,排除了数值溢出的可能性。
	\[
	l(y^{(i)}, \hat{y}^{(i)}) = -\log \hat{y}_j = -\log \frac{\exp(z_j)}{\sum_{k=1}^q \exp(z_k)} = \log \sum_{k=1}^q \exp(z_k) - z_j
	\]\\
	
	\textbf{4、返回概率最大的分类标签总是最优解吗?例如,医疗诊断场景下可以这样做吗?}
	
	\noindent \textbf{解}:返回概率最大的分类标签并不总是最优解。
	
	概率最大的分类标签只是基于模型的预测结果,可能存在误差。模型的预测结果可能受到多种因素的影响,如数据质量、特征选择、模型选择等。因此,仅仅依靠概率最大的分类标签可能会导致错误的判断。
	
	在医疗诊断等领域,决策可能需要综合考虑多个因素,如患者的病史、症状、实验室检查结果等。仅仅依靠概率最大的分类标签可能无法充分考虑这些因素,导致不准确的诊断结果。概率最大的分类标签可以作为参考,但不能作为唯一的依据。\\
	
	\textbf{5、假设我们使用 softmax 回归来预测下一个单词,可选取的单词数目过多可能会带来哪些问题?}
	
	\noindent \textbf{解}:1)计算复杂度增加:softmax 回归的计算复杂度与可选取的单词数目成正比。当可选取的单词数目过多时,计算 softmax 函数的指数项和可能会变得非常耗时,导致模型训练和推理的效率下降。
	
	2)参数空间增大:softmax 回归的参数矩阵的大小与可选取的单词数目成正比。当可选取的单词数目过多时,模型需要学习更多的参数,导致模型的参数空间变得非常庞大,增加了模型训练的难度和模型的存储需求。
	
	3)数据稀疏性增加:当可选取的单词数目过多时,训练数据中每个单词的出现频率可能会变得非常低,导致数据的稀疏性增加。这会使得模型难以准确地估计每个单词的概率分布,可能导致模型的预测性能下降。
	
	4)模型泛化能力下降:当可选取的单词数目过多时,模型可能会过于依赖训练数据中出现频率较高的单词,而忽略了其他单词的特征。这会导致模型的泛化能力下降,对于训练数据中未出现或出现频率较低的单词的预测效果较差。
	
	因此,在使用 softmax 回归进行下一个单词的预测时,需要权衡可选取的单词数目,选择适当的单词数量以平衡计算复杂度、参数空间、数据稀疏性和模型泛化能力。
	
\end{document}