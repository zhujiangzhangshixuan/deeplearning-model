\documentclass[UTF8]{ctexart}
\usepackage[textwidth=444bp,vmargin=2.5cm]{geometry}
%\usepackage{abstract} %生成摘要使用的宏包
\usepackage[colorlinks,linkcolor=black,anchorcolor=black,citecolor=black]{hyperref} %“colorlinks”的意思是将超链接以颜色来标识,而并非使用默认的方框来标识。linkcolor,anchorcolor, citecolor分别表示用来标识link, anchor, cite等各种链接的颜色。此处我们均设为黑色。
\usepackage{fancyhdr} %插入页脚的宏包
\usepackage{graphicx}%图片宏包
\usepackage{float}
\graphicspath{{./figure/}}
\DeclareGraphicsExtensions{.pdf,.jpeg,.png,.jpg}%在添加图片后只需要图片的名字,而不需要拓展名
\usepackage{subfigure}%并列插入图片
\usepackage{amsmath}%公式宏包
\usepackage[T1]{fontenc}% 统一修改正文和数学字体为Adobe Utopia, 这个字体和Times有些像
\usepackage{newtxtext, newtxmath}  %两种使用Times New Roman 字体的方法
\linespread{1.5}%设置行间距为1.5倍行距
\renewcommand{\abstractname}{\Large\textbf{摘要}}
\usepackage{appendix}%加入附录需使用appendix宏包
\usepackage{listings}%插入代码
\usepackage{xcolor}

\usepackage{booktabs}
\usepackage{longtable}
\usepackage{array}
\usepackage{multirow}
\usepackage{makecell}
\usepackage{bm}
\usepackage{pythonhighlight}
\usepackage{ctex}
\usepackage[linesnumbered,ruled]{algorithm2e}
\ctexset{
	% 修改 section。
	section={   
		name={,、},
		number={\chinese{section}}
	}
}
\ctexset{
	% 修改 section。
	section={   
		name={,、},
		number={\chinese{section}},
		aftername=\hspace{1pt}
	}
}
\usepackage{enumitem}
\usepackage{tabularx}
%\usepackage{setspace}%使用间距宏包

\begin{document}    %文档的开始,一定要有文档的结束,才能生效
	\setlength{\abovedisplayskip}{2pt}
	\setlength{\belowdisplayskip}{2pt}
	\setlength{\abovedisplayshortskip}{2pt}
	\setlength{\belowdisplayshortskip}{2pt}
	%------------------------标题-------------------------
	%\begin{titlepage}%使页码跳过这页
	\begin{center}
		\heiti\zihao{3}\textbf{3.1 \, 线性回归重点摘录与练习解答} %标题3号加粗
		\vspace{2ex}
	\end{center}
	
	
	%\end{titlepage}
	%-------------------------正文部分-------------------------
	%修改页眉页脚
	\pagestyle{fancy}
	\lhead{}
	\chead{}
	\rhead{}
	%\lfoot{}
	\cfoot{\thepage}
	%\rfoot{} %空格即表示空白
	\renewcommand{\headrulewidth}{0pt}
	\renewcommand{\footrulewidth}{0pt} %设置页眉页脚分割线的宽度,如果为0pt,则不显示线条
	
	\textbf{(1) 线性模型}
	
	对于高维数据集,建模时采用线性代数表示法会比较方便。当我们的输入包含 $d$ 个特征时,我们将预测结果 $\hat{y}$ (通常使用 “尖角” 符号表示 $y$ 的估计值)表示为:
	$$
	\hat{y} = w_1 x_1 + \ldots + w_d x_d + b.
	$$
	
	将所有特征放到向量 $\bm{x} \in \mathbb{R}^d$ 中,并将所有权重放到向量 $\bm{w} \in \mathbb{R}^d$ 中,我们可以用点积形式来简洁地表达模型:
	$$
	\hat{y} = \bm{w}^\top \bm{x} + b.
	$$
	
	对于 $y$ 也是高维的情况,设特征集合 $\bm{X}$,预测值 $\hat{\bm{y}} \in \bm{R}^n$ 可以通过矩阵-向量乘法表示为:
	$$
	\hat{\bm{y}} = \bm{X} \bm{w} + b
	$$
	这个过程中的求和将使用在2.1.3节中有详细介绍的广播机制。给定训练数据特征 $\bm{X}$ 和对应的已知标签 $\bm{y}$,线性回归的目标是找到一组权重向量 $\bm{w}$ 和偏置 $b$:当给定从 $\bm{X}$ 的同分布中取样的新样本特征时,这组权重向量和偏置能够使得新样本预测标签的误差尽可能小。
	
	\textbf{(2) 平方损失函数}
	
	当样本 $ i $ 的预测值为 $\hat{y}^{(i)}$,其相应的真实标签为 $ y^{(i)} $ 时,平方误差可以定义为以下公式:
    \[
    l^{(i)}(\bm{w}, b) = \frac{1}{2} \left( \hat{y}^{(i)} - y^{(i)} \right)^2.
    \]
    
    为了度量模型在整个数据集上的质量,我们需要计算在训练集 $ n $ 个样本上的损失均值(也等价于求和):
    $$
    L(\bm{w}, b) = \frac{1}{n} \sum_{i=1}^{n} l^{(i)}(\bm{w}, b) = \frac{1}{n} \sum_{i=1}^{n} \frac{1}{2} \left( \bm{w}^\top \bm{x}^{(i)} + b - y^{(i)} \right)^2.
    $$
	
	在训练模型时,我们希望寻找一组参数 $(\bm{w}^*, b^*)$,这组参数能最小化在所有训练样本上的总损失。如下式:
	$$
	\bm{w}^*, b^* = \arg \min_{\bm{w}, b} L(\bm{w}, b).
	$$
	
	\textbf{(3) 解析解}
	
	我们的预测问题是最小化 $\|\bm{y} - \bm{X}\bm{w}\|^2$,将损失关于 $\bm{w}$ 的导数设为0,即为 $\displaystyle \frac{\partial l}{\partial \bm{w}} = 0$ 时,
	$$
	\displaystyle
	\frac{\partial \| \bm{y} - \bm{X} \bm{w} \|^2}{\partial \bm{w}}
	= \frac{\partial \frac{1}{2} (\bm{y} - \bm{X} \bm{w})^2}{\partial \bm{w}}
	= (\bm{y} - \bm{X} \bm{w}^*) (-\bm{X}^\top) = 0
	$$
	得到解析解:
	$$
	\displaystyle
	\bm{w}^* = (\bm{X}^\top \bm{X})^{-1} \bm{X}^\top \bm{y}.
	$$
	
	\textbf{(4) 随机梯度下降}
	
	随机度下降最简单的用法是计算损失函数(数据集中所有样本的损失均值)关于模型参数的导数(在这里也可以称为梯度)。但实际中的执行可能会非常慢:因为在每一次更新参数之前,我们必须遍历整个数据集。因此,我们通常会在每次需要计算更新的时候随机抽取一小批样本,这种变体叫做小批量随机梯度下降(minibatch stochastic gradient descent)。
	
	在每次迭代中,我们首先随机抽样一个小批量 $\mathcal{B}$,它是由固定数量的训练样本组成的。然后,我们计算小批量的平均损失关于模型参数的导数(也可以称为梯度)。最后,我们将梯度乘以一个预先确定的正数 $\eta$,并从当前参数的值中减掉。
	$$
	(\bm{w}, b) \leftarrow (\bm{w}, b) - \frac{\eta}{|\mathcal{B}|} \sum_{i \in \mathcal{B}} \partial_{(\bm{w}, b)} l^{(i)}(\bm{w}, b).
	$$
	总结一下可知算法的步骤如下:
	\begin{enumerate}
		\item 初始化模型参数的值,如随机初始化;
		\item 从数据集中随机抽取小批量样本且在负梯度的方向上更新参数,并不断迭代这一步骤。对于平方损失和仿射变换。
	\end{enumerate}
	我们可以明确地写成如下形式:
	$$
	\bm{w} \leftarrow \bm{w} - \frac{\eta}{|\mathcal{B}|} \sum_{i \in \mathcal{B}} \partial_{\bm{w}} l^{(i)}(\bm{w}, b) = \bm{w} - \frac{\eta}{|\mathcal{B}|} \sum_{i \in \mathcal{B}} \bm{x}^{(i)} \left( \bm{w}^\top \bm{x}^{(i)} + b - y^{(i)} \right),
	$$
	
	$$
	b \leftarrow b - \frac{\eta}{|\mathcal{B}|} \sum_{i \in \mathcal{B}} \partial_b l^{(i)}(\bm{w}, b) = b - \frac{\eta}{|\mathcal{B}|} \sum_{i \in \mathcal{B}} \left( \bm{w}^\top \bm{x}^{(i)} + b - y^{(i)} \right).
	$$
	
	
	\, 
	
	
	\textbf{(5) 问题解答}
	
	\textbf{1、假设有一些数据$x_1, \ldots, x_n \in \mathbb{R}$。目标是找到一个常数$b$,使得最小化$\sum_i (x_i - b)^2$,找到最优值$b$的解析解,这个问题及其解与正态分布有什么关系?}
	
	\noindent {\textbf{解}:即求
	\[
	b^* = \arg \min \sum_i (x_i - b)^2
	\] 
	
	令
	\[
	\frac{\partial \sum_i (x_i - b)^2}{\partial b} = 0
	\]
	
	即 $2 \sum_i (x_i - b^*) = 0$,故
	\[
	 b^* = \frac{1}{n} \sum_i x_i.
	\]
    }
	
	我们先求取样本关于参数 $b$ 的极大似然估计,令 $ x_i = b + \epsilon $,其中$\epsilon \sim \mathcal{N}(0, \sigma^2)$,则似然函数为:
	\[
	L(x \mid b) = (2\pi\sigma^2)^{-\frac{n}{2}} \exp\left( -\frac{1}{2\sigma^2} \sum_{i=1}^n (x_i - b)^2 \right)
	\]
	
	因此对数似然函数为:
	\[
	-l(x \mid b)= -\log L(x \mid b) = \frac{n}{2} \log(2\pi\sigma^2) + \sum_{i=1}^n \frac{1}{2\sigma^2} (x_i - b)^2
	\]
	
	若求 $\arg \max_b L(x \mid b)$,即求 $\arg \min_b -l(x \mid b)$.根据数理统计写出似然方程可知:
	\[
	b^* = \frac{1}{n} \sum_{i=1}^n x_i.
	\]
	
	也即求上一问题的解析解。因此,在高斯噪声的假设下,最小化均方误差等价于对线性模型的极大似然估计。
	
	
	\textbf{2、推导出使用平方误差的线性回归优化问题的解析解。为了简化问题,可以忽略偏置$b$(我们可以通过向$\mathbf X$添加所有值为1的一列来做到这一点)。}
		
	\noindent {\textbf{解}}:这里只给出部分问题的解答,首先对用矩阵和向量表示法写出优化问题(将所有数据视为单个矩阵,将所有目标值视为单个向量)。
	$$
	\hat{\bm{Y}}_{n,q} = \bm{X}_{n,d+1} \bm{w}_{d+1,q}
	$$
	
	然后是计算损失对 $ w $ 的梯度:
	$$
	L = \frac{1}{2} (\bm{Y} - \hat{\bm{Y}})^2
	$$
	$$
	\frac{\partial L}{\partial \bm{w}} = \frac{\partial \frac{1}{2} (\bm{Y} - \bm{X} \bm{w})^2}{\partial \bm{w}} = (\bm{Y} - \bm{X} \bm{w})(-\bm{X}^\top)
	$$
	
	第三个小问是通过将梯度设为 0、求解矩阵方程来找到解析解。
	$$
	(\bm{Y} - \bm{X} \bm{w})(-\bm{X}^\top) = 0
	$$
	$$
	-\bm{X}^\top \bm{Y} + \bm{X}^\top \bm{X} \bm{w} = 0
	$$
	
	可得解析解表达式为:
	$$
	\bm{w}^* = (\bm{X}^\top \bm{X})^{-1} \bm{X}^\top \bm{Y}
	$$
	
	解析解可能比使用随机梯度下降(SGD)更好的情况包括:
	\begin{enumerate}
		\item 简单问题:解析解通常适用于简单的问题,其中目标函数和约束条件很容易求导并求解。在这种情况下,直接计算解析解比使用SGD更高效。
		\item 小规模数据集:对于小规模的数据集,计算解析解可以很快完成,并且由于数据量较小,解析解的计算开销相对较小。
		\item 显式公式要求:某些应用场景可能要求得到显式的公式解析解,例如需要解释、推导或证明的问题。
	\end{enumerate}
	
	然而,解析解的方法在以下情况下可能会失效:
	\begin{enumerate}
		\item 复杂问题:对于复杂的问题,目标函数和约束条件可能很难求导或求解,或者求解过程可能非常复杂甚至不存在解析解。在这种情况下,使用SGD等数值优化算法可能更适合。
		\item 大规模数据集:对于大规模数据集,计算解析解的计算复杂度可能非常高,甚至无法完成。在这种情况下,SGD通常更具可行性和可扩展性。
		\item 随机性和噪声:如果目标函数存在随机性或噪声,并且我们希望在优化过程中考虑到这些因素,那么SGD等迭代方法通常更合适,因为它们可以根据采样的随机梯度进行逐步的调整。
	\end{enumerate}

	%正文中引用到参考文献的地方\cite{01} \cite{02}
	%-------------------------附录部分-------------------------
	% 使用\begin{appendices} \end{appendices} 或者直接用\appendix
	\appendix
	%代码格式设置,代码的设置与具体的编程语言有关,比赛时上网搜素即可
	\definecolor{dkgreen}{rgb}{0,0.6,0}
	\definecolor{gray}{rgb}{0.5,0.5,0.5}
	\definecolor{mauve}{rgb}{0.58,0,0.82}
	\definecolor{mydarkblue}{RGB}{0, 0, 128} % 示例:深蓝色,RGB值为(0, 0, 128) 
	\definecolor{codegreen}{rgb}{0,0.6,0}
	\definecolor{codegray}{rgb}{0.5,0.5,0.5}
	\definecolor{codepurple}{rgb}{0.58,0,0.82}
	\definecolor{backcolour}{rgb}{0.95,0.95,0.92}
	\lstset{ %
		language=Python,                % the language of the code
		basicstyle=\footnotesize,           % the size of the fonts that are used for the code
		numbers=left,                   % where to put the line-numbers
		numberstyle=\tiny\color{gray},  % the style that is used for the line-numbers
		stepnumber=2,                   % the step between two line-numbers. If it's 1, each line 
		% will be numbered
		numbersep=5pt,                  % how far the line-numbers are from the code
		backgroundcolor=\color{white},      % choose the background color. You must add \usepackage{color}
		showspaces=false,               % show spaces adding particular underscores
		showstringspaces=false,         % underline spaces within strings
		showtabs=false,                 % show tabs within strings adding particular underscores
		frame=single,                   % adds a frame around the code
		rulecolor=\color{black},        % if not set, the frame-color may be changed on line-breaks within not-black text (e.g. commens (green here))
		tabsize=2,                      % sets default tabsize to 2 spaces
		captionpos=b,                   % sets the caption-position to bottom
		breaklines=true,                % sets automatic line breaking
		breakatwhitespace=false,        % sets if automatic breaks should only happen at whitespace
		title=\lstname,                   % show the filename of files included with \lstinputlisting;
		% also try caption instead of title
		keywordstyle=\color{blue},          % keyword style
		commentstyle=\color{codegreen},       % comment style
		stringstyle=\color{codepurple},         % string literal style
		escapeinside={\%*}{*)},            % if you want to add LaTeX within your code
		morekeywords={*,...}               % if you want to add more keywords to the set
	}
	
	
	


\end{document}